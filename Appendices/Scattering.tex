\chapter{Scattering simulation}\label{app:scattering}
\lhead{\emph{Appendix: Scattering simulation}}
Much of our statistical analysis is based on simulated data consisting of nuclear recoils. The observed distribution of recoil energies should follow the differential event rate $\textrm{d}R/\textrm{d}E_r$. For simple analytic event rates it is simple to sample random energies from this distribution directly, however for more complicated distributions for instance those that also depend upon recoil direction and time, it is more efficient to simulate the scattering process. This is particularly true when generating recoil energies and directions in the laboratory coordinate system (North, West, zenith). Because the lab velocity $\textbf{v}_\textrm{lab}$ and Solar vector $\hat{\textbf{q}}_\odot$ both have two separate modulation periods of 24 hours and 1 year in length, it is more efficient to sample from a 1-dimensional distribution and simulate the stochastic scattering process than sample from the 4-dimensional distribution which requires a time resolution of less than $\mathcal{O}({\rm hr})$ in size to be accurate. We detail these calculations for WIMPs and neutrinos below. 

Simulating WIMP events first envolves calculating the total rate as a function of time, this should then be used to provide event times which follow the annual and daily modulations of the WIMP signal. Then for each event time we generate a WIMP velocity. Since on Earth we observe the velocity {\it flux} distribution $v f(\textbf{v}~+~\textbf{v}_\textrm{lab}(t))$ (i.e. we encounter faster particles more frequently than slower ones), we sample velocities from this. For isotropic velocity distributions one can first sample from the corresponding speed distribution, then generate a random Galactic frame angle to create a velocity. For anisotropic distributions such as streams one must sample from the 3-dimensional velocity flux distribution.

Once a velocity (with its event time) has been generated it can then be rotated into the laboratory frame if desired. Then the energy of its recoil is given by,
\begin{equation}
	E = \frac{2 \mu_{\chi p}^2 v^2 \cos^2\theta}{m_N},
\end{equation}
where $\theta$ is the scattering angle between the initial WIMP direction and the direction of the recoiling nucleus. The scattering angle is related to the centre of mass scattering angle, $\theta_\textrm{com}$, by
\begin{equation}
\theta = \frac{\pi}{2} - \frac{\theta_\textrm{com}}{2}.
\end{equation}
The scattering process is isotropic in the centre of mass frame so the centre of mass angle is taken to be isotropically distributed, i.e. $\theta_\textrm{com}= \cos^{-1} (2u-1)$ where $u \in [0,1)$ is a uniformly random variate. The recoil vector is generated by deflecting the initial WIMP direction by the elevation angle $\theta$ and then rotating the deflected vector by a uniformly randomly generated angle $\phi \in [0,2\pi)$ around the initial WIMP direction.

The correction due to the nuclear form factor is not carried out on an event-by-event basis as it is defined as a correction to the overall recoil energy spectrum. Instead it is taken into account by calculating $F^2(E_r)$ for each recoil and then discarding each recoil with a probability $1-F^2(E_r)$. 

For neutrinos the initial process is slightly different. As before we first must generate neutrino event times according to the desired annual modulation signal. To generate neutrino events we then select neutrino energies from the neutrino flux $E^2_\nu \, \textrm{d}\Phi_\nu/\textrm{d}E_\nu$. For Solar neutrinos the incident direction is given by $\hat{\textbf{q}}_\odot$ whereas for atmospheric neutrinos we either select a zenith angle from the FLUKA results~\cite{Battistoni:2005pd} or for later results in which we ignore this angular distribution we select from an isotropic distribution as with the DSNB events. The scattering simulation then proceeds as with the case of WIMPs. 

For both WIMP and neutrino events we generate a number of events from a Poisson distribution with a mean $N_{\rm exp}$ which is found from the total event rate multiplied by exposure. For neutrinos there is an added step involved with accounting for the flux uncertainty. Before calculating the expected number of events we must first select a total flux normalisation from a Gaussian distibution with mean values and widths listed in Table~\ref{tab:neutrino}. 