\chapter{Conclusions}
\lhead{\emph{Conclusions}}
Advances in precision cosmology have brought about compelling evidence for the darkness of the matter content in the Universe. The unmasking of this dark matter by one of a range of complementary experimental strategies will most certainly herald a new era for particle physics. If terrestrial direct detection experiments searching for the interaction between dark matter and nuclei are successful, this will not only uncover the particle identity of dark matter, but it will also unlock the potential for the galactic archaeology of the Milky Way. As discussed in this thesis, understanding the structure of our dark matter halo is crucial for achieving the particle physics goals of dark matter detection, for example measuring the properties of the dark matter particle and dealing with the neutrino background. But experiments on Earth are also a unique and unmatched probe of dark matter halos on astronomically inaccessible scales. As the nature of dark matter remains unknown, we have tried to translate the problems faced in detecting dark matter, and their subsequent solutions, to two of the most popular candidate particles namely weakly interacting massive particles (WIMPs) and axions. 

In the framework of the WIMP direct detection we have predominantly focused on the prospects for future directional detectors. We hope that the results presented in this thesis make an impactful physics case and motivation for the development and construction of large directionally sensitive experiments in the future. For instance, as we explored in Chapter~\ref{chapter:directional}, directional detectors in principle outclass conventional approaches when attempting to uncover substructure in the local velocity distribution. For tidal streams from the stripping of nearby satellite galaxies, the detection of the canonical example from the Sagittarius dwarf would need $\mathcal{O}(300)$ events to make a conclusive detection. We also compared model dependent and independent methods for reconstructing the full 3-dimensional structure of the velocity distribution, a feat that would be incredibly difficult in non-directional experiments. We showed that one can probe the level of substructure in the local distribution with an empirical binned parameterisation, that assumes absolutely no knowledge about the functional form of the underlying distributions. Trying to understand the structure of dark matter halos, and uncovering the accretion history of our own Milky Way, will be preeminent goals of a post-discovery era. To this end, we believe we have made a compelling case for directional detectors.

While directional detectors present theoretically fascinating prospects for detecting dark matter directly, there are immediate and severe caveats in its experimental implementation. A range of prototype experiments already exist, and although these are promising, attempting to measure directionality introduces many additional complications. The issues we have discussed here: obtaining good angular resolution, forward-backward sense recognition, and the reconstruction of the full three dimensions of a recoil track, will all limit the discovery reach and physics potential of an experiment in practice. We showed in the context of the neutrino background, the prospects are good even with non-ideal detectors. As long as angular resolution better than around $30^\circ$ can be achieved then the background due to Solar neutrinos can easily be overcome. In the case of atmospheric and diffuse supernova neutrinos we also require sense recognition, but in this case the benefit of directionality in general is not as significant. We would also desire an experiment with full 3-dimensional readout, but we have demonstrated that this is not absolutely essential and good progress can be made through the neutrino floor even without it. However for doing WIMP astronomy some of these caveats, like sense recognition, may be too severe. In future work on the subject of the discovery reach for directional detectors we must continue to keep these in mind. In particular, directional detectors that compromise on the full reconstruction in 3-dimensions in favour of an enhanced ability to scale to large exposures, may in the end turn out to be the most feasible and powerful approach for detecting dark matter. 

Arguably the next forseeable roadblock in the progression of direct dark matter detection will be dealing with the neutrino floor. This in itself is an exciting prospect as coherent neutrino-nucleus scattering remains an unobserved interaction of the standard model and there are potentially many interesting questions relating to neutrino physics that may be answered with a new mode of detection. In Chapter~\ref{chapter:nufloor} we discussed the problem that the neutrino background presents to dark matter detection. We showed that once the aforementioned astrophysical uncertainties are embedded in the calculation of the neutrino floor, the problem is made much more severe. Indeed, the neutrino background not only prohibits WIMP models from being `discovered', the measurement of WIMP properties around and even above this limit are greatly inhibited by the background as well. Clearly if we wish to continue the search for WIMPs below the floor we must devise new strategies to subtract the background. We studied in detail the excellent prospects presented by directional detectors. But in the future we will also be able to consider the interplay between neutrino telescopes and Solar model building in improving the uncertainty on the neutrino background. 

The subject of dark matter astronomy with direct detection experiments allowed us to transition into a study of an alternative particle candidate: axions, and the generalisation axion-like particles (ALPs). We focused here on microwave cavity haloscope experiments which are designed to resonantly detect the Primakoff conversion of axions into photons. We showed that the prospects for axion astronomy are intriguing. We found that one could measure astrophysical parameters to a level of precision that could far exceed that of even directional detectors in the case of WIMPs, even in existing experiments. This is true for basic underlying parameters such as the Solar peculiar velocity, but could also be applied to tidal streams from the stripping of nearby dwarf galaxies, as well as those from the disruption of axion miniclusters. The caveat to these claims is that they work principally on the assumption that the axion has already been successfully detected in a haloscope experiment like ADMX. This was of course true for many of our results in the context of WIMPs, but the disadvantage of axion detection is that the enormous available parameter space requires a substantial range of different and complex experimental strategies. In the future if these questions become more relevant, for example if the existence of axions or ALPs is suggested in the lab or through astrophysical observation, then we may need to develop slightly different strategies to measure the velocity distribution. It may also be the case that in other axion dark matter detection proposals such as with dielectric disks or dish antennae, or nuclear magnetic resonance experiments for couplings to nuclei, then the great prospects for axion astronomy may be shared with the case study we presented here.

In any eventuality we all hope that the detection of dark matter is imminent. At this point one or more of the issues we have raised here will become of direct and immediate importance. For this reason we hope that this thesis may be useful to others in the future once we have emerged in an era of dark matter detection.


